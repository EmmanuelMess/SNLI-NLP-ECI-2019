\documentclass{article}

\usepackage[utf8x]{inputenc}%%permite acentos y otras bobadas
\usepackage[T1]{fontenc}%%hace que letras con acento sean una sola

\title{Escuela de las Ciencias Informáticas 2019\\
	{\small{Curso M1}}\\
	Procesamiento del lenguaje natural mediante redes neuronales\\
	{\small{Trabajo practico: replicación de los resultados del paper}}\\
	"Annotation Artifacts in Natural Language Inference Data"\footnote{https://www.aclweb.org/anthology/N18-2017}}
\author{Facundo Emmanuel Messulam\footnote{Facultad de Ciencias Exactas, Ingeniería y Agrimensura, Licenciatura en Ciencias de la Computación}
	\and
	Ramiro Hernán Gatti\footnote{Instituto de Investigación y Desarrollo en Bioingeniería y Bioinformática, CONICET-UNER}}

\begin{document}
    \begin{titlepage}
        \maketitle
        \thispagestyle{empty}
    \end{titlepage}
	
	\section*{Preliminar}
	El paper propone el uso de "fasttext" el clasificador lineal de Facebook\cite{joulin2017bag}. Que usamos para
	
	\newpage
	\begin{thebibliography}{9}
		\bibitem{joulin2017bag} 
		A. Joulin, E. Grave, P. Bojanowski, T. Mikolov. \textit{Bag of Tricks for Efficient Text Classification}. Association for Computational Linguistics, 2017
	\end{thebibliography}
\end{document}
